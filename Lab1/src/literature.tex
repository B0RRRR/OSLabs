\section{Исходная программа}


\subsection*{Файл parent.c}

\begin{lstlisting}[language=C,caption={Код программы родительского процесса},captionpos=b]
#include <stdio.h>
#include <stdlib.h>
#include <unistd.h>
#include <sys/wait.h>
#include <fcntl.h>
#include <string.h>
#include <errno.h>

int main() {
    char filename[256];
    // fd for parent -> children
    int pipefd[2];
    pid_t pid;

    if (pipe(pipefd) == -1) {
        fprintf(stderr, "Failed to create pipe\n");
        return -1;
    }

    printf("Filename:\n");
    if (scanf("%255s", filename) != 1) {
        fprintf(stderr, "Incorrect file name\n");
        return -1;
    }

    int fd_file = open(filename, O_RDONLY);
    if (fd_file == -1) {
        fprintf(stderr, "Failed to open file\n");
        return -1;
    }

    pid = fork();
    if (pid < 0) {
        fprintf(stderr, "Fork failed\n");
        close(fd_file);
        close(pipefd[0]);
        close(pipefd[1]);
        return -1;
    }

    if (pid == 0) {
        // Close pip for reading
        close(pipefd[0]);

        // input -> file
        if (dup2(fd_file, STDIN_FILENO) == -1) {
            fprintf(stderr, "dup2(fd_file->stdin) failed:\n");
            return -1;
        }

        // child proc -> pipe
        if (dup2(pipefd[1], STDOUT_FILENO) == -1) {
            fprintf(stderr, "dup2(pipe_write->stdout) failed:\n");
            return -1;
        }

        close(fd_file);
        close(pipefd[1]);

        // proc -> child
        char *argv[] = {"./child", NULL};
        execv(argv[0], argv);

        fprintf(stderr, "execv failed:\n");
        return -1;
    } else {
        // close fd for writing
        close(pipefd[1]);
        close(fd_file);

        // Any answer in terminal
        ssize_t n;
        char buffer[4096];
        while ((n = read(pipefd[0], buffer, sizeof(buffer))) > 0) {
            ssize_t out = 0;
            while (out < n) {
                ssize_t w = write(STDOUT_FILENO, buffer + out, n - out);
                if (w == -1) {
                    perror("write");
                    break;
                }
                out += w;
            }
        }
          
    if (n == -1) {
        fprintf(stderr, "Failed read from pipe");
    }

    close(pipefd[0]);
    if (wait(NULL) == -1) {
        perror("wait");
    }

    }

    return 0;
}
\end{lstlisting}

\subsection*{Файл child.c}

\begin{lstlisting}[language=C,caption={Код программы дочернего процесса},captionpos=b]
#include <stdio.h>
#include <stdlib.h>

int main() {
    float x, sum = 0.0f;
    int read_count = 0;   // count numbers in str
    int c;

    while (1) {
        int r = scanf("%f", &x);

        if (r == 1) {
            sum += x;
            read_count++;
        } else if (r == EOF) {
            // end of file
            if (read_count > 0) {
                printf("Sum: %.2f\n", sum);
            }
            break;
        } else {
            /* scanf cant read number */
            printf("Not number in file!\n");
            return -1;
        }

        // If next tab, space or n
        c = getchar();
        if (c == '\n' || c == EOF) {
            if (read_count > 0) {
                printf("Sum: %.2f\n", sum);
                sum = 0.0f;
                read_count = 0;
            }
            if (c == EOF) break;
        } else if (c == ' ' || c == '\t' || c == '\r') {
            // Correct symbols
        } else {
            // Incorrect Symbol
            printf("Error: Incorrect Symbol '%c'\n", c);
            return -1;
        }
    }

    return 0;
}
\end{lstlisting}

\section{Вывод strace}

\begin{verbatim}
execve("./parent", ["./parent"], 0x7fff77b42d10) = 0
pipe2([3, 4], 0) = 0
write(1, "Filename:\n", 10) = 10
read(0, "input.txt\n", 1024) = 10
openat(AT_FDCWD, "input.txt", O_RDONLY) = 5
clone(...) = 6474
read(3, "Sum: 3354.30\n", 4096) = 13
write(1, "Sum: 3354.30\n", 13) = 13
--- SIGCHLD --- 
wait4(-1, NULL, 0, NULL) = 6474
\end{verbatim}