\section{Результаты}

Разработанное решение состоит из двух программ на языке C: \texttt{parent.c} и \texttt{child.c}, 
которые взаимодействуют между собой с использованием механизмов межпроцессного взаимодействия (IPC) в операционной системе Linux.

\subsection*{Ключевые особенности}
\begin{enumerate}
    \item Родительский процесс отвечает за управление (создание канала, запуск дочернего процесса, получение и вывод результата), а дочерний процесс выполняет исходную задачу.
    \item Стандартный поток вывода дочернего процесса перенаправляется в канал, из которого читает родитель.
    \item С помощью системного вызова \texttt{dup2} стандартный ввод дочернего процесса заменяется файлом, выбранным пользователем, а стандартный вывод перенаправляется в канал.
    \item Родительский и дочерний процесс реализованы как разные исполняемые файлы.
\end{enumerate}
