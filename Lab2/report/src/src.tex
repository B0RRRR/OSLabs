\section{Метод решения}
Задача заключается в многопоточной обработке матрицы вещественных чисел с применением фильтров эрозии и наращивания. 
Программа распределяет строки матрицы между потоками и измеряет время выполнения для различных чисел потоков.

\subsection*{Архитектура программы}
\dirtree{%
.1 Lab2/. 
.2 inc/.
.3 filter\_worker.h.
.3 timer.h.
.3 matrix\_filters.h.
.3 matrix\_utils.h.
.2 src/.
.3 filter\_worker.c.
.3 main.c.
.3 matrix\_filters.c.
.3 matrix\_utils.c.
.2 Makefile.
}
\vspace{5mm}

\section{Описание программы}

\subsection*{Файл \texttt{main.c}}
\textbf{Назначение:}  
Главный модуль программы, отвечающий за инициализацию и запуск основного алгоритма.

\begin{itemize}
    \item Считывает аргументы командной строки: размеры матрицы $M$ и $N$, количество повторов фильтров $K$, а также максимальное число одновременно работающих потоков $T_{max}$.
    \item Передаёт эти параметры функции \texttt{run\_matrix\_filters()}, определённой в модуле \texttt{matrix\_filters.c}.
    \item Возвращает код завершения программы.
\end{itemize}

---

\subsection*{Файл \texttt{matrix\_filters.c}}
\textbf{Основная логика программы:}
\begin{itemize}
    \item Реализует функцию \texttt{run\_matrix\_filters()}, выполняющую весь процесс вычислений.
    \item Инициализирует исходную матрицу случайными значениями.
    \item Запускает цикл по количеству потоков: $1, 2, 4, \ldots, T_{max}$.
    \begin{itemize}
        \item Делит матрицу по строкам между потоками.
        \item Создаёт заданное количество потоков.
        \item В каждом потоке вызывает \texttt{apply\_filter\_static()} для обработки выделенного диапазона строк.
        \item После завершения всех потоков измеряет время выполнения.
        \item Записывает результаты в файл \texttt{results.csv} для последующего построения графика зависимости времени от числа потоков.
    \end{itemize}
\end{itemize}

---

\subsection*{Файл \texttt{filter\_worker.c}}
\textbf{Функции и логика:}
\begin{itemize}
    \item \texttt{apply\_filter\_static()} — функция, выполняемая каждым потоком. Применяет фильтры эрозии и наращивания $K$ раз для своего диапазона строк.
    \item \texttt{min\_in\_3x3()} и \texttt{max\_in\_3x3()} — вычисляют минимальные и максимальные значения в окне $3 \times 3$ вокруг каждого элемента матрицы.
\end{itemize}

\textbf{Структуры данных:}
\begin{itemize}
    \item \texttt{ThreadWork} — структура, содержащая указатели на исходную и результирующую матрицы, количество итераций фильтрации и диапазон строк, обрабатываемых данным потоком.
\end{itemize}

---

\subsection*{Файл \texttt{matrix\_utils.c}}
\textbf{Функции:}
\begin{itemize}
    \item \texttt{create\_matrix()} — выделяет память под матрицу заданного размера.
    \item \texttt{fill\_matrix()} — заполняет матрицу случайными вещественными числами.
    \item \texttt{free\_matrix()} — освобождает выделенную память.
\end{itemize}

