\section{Выводы}
В ходе выполнения лабораторной работы были приобретены практические навыки в организации многопоточной обработки данных в операционных системах Linux, а также в безопасной работе с разделяемыми ресурсами при распараллеливании вычислений.\\
Была разработана и отлажена программа на языке C, реализующая многократное применение фильтров эрозии и наращивания на матрице вещественных чисел. Программа использует стандартные средства многопоточности POSIX (\texttt{pthread\_t}), что обеспечивает корректное выполнение на системах семейства Unix и позволяет гибко контролировать количество потоков через аргумент командной строки.\\
В результате работы программа запускает указанное пользователем максимальное число потоков, при этом работа распределяется статически: каждый поток обрабатывает отдельный диапазон строк матрицы. Это минимизирует накладные расходы на синхронизацию и предотвращает гонки данных.\\
Были обработаны возможные ошибки ввода аргументов командной строки и обеспечена корректная инициализация и завершение всех потоков. Экспериментально подтверждена эффективность параллельных вычислений: время выполнения сокращается с увеличением числа потоков до аппаратного предела (числа логических ядер процессора), что подтверждается построенным графиком зависимости времени от количества потоков.\\
Результаты работы демонстрируют, что статическое распределение задач между потоками позволяет достичь высокой производительности и масштабируемости алгоритма при обработке больших матриц.
\pagebreak
