\section{Выводы}
В ходе выполнения лабораторной работы была разработана программа на языке C,
демонстрирующая механизм взаимодействия между двумя процессами с использованием POSIX разделяемой памяти и системных сигналов.
В результате был реализован надёжный протокол синхронизации, в котором родительский процесс передаёт дочернему строковые команды,
содержащие вещественные числа, а дочерний процесс выполняет их обработку и возвращает результат обратно через общий сегмент памяти. Работоспособность решения подтверждена корректной передачей данных,
отсутствием гонок и успешной обработкой ошибок системных вызовов.
В процессе работы были изучены методы создания процессов,
установки обработчиков сигналов, организации совместного доступа к памяти и синхронизации исполнения. Реализация была структурирована по каталогам \texttt{inc/} и \texttt{src/},
что позволило обеспечить ясное разделение логики между модулями программы и упростить дальнейшее развитие проекта.
Полученный результат показывает, что связка сигналов и разделяемой памяти является эффективным средством межпроцессного взаимодействия для задач подобного типа,
обеспечивая высокую скорость обмена и минимальные накладные расходы.
