\section{Метод решения}

Задача заключается в межпроцессном взаимодействии двух процессов (родительского и дочернего) с использованием разделяемой памяти, сигналов и простого протокола синхронизации.  
Родительский процесс читает команды, передаёт их дочернему через общий сегмент памяти и ожидает ответа.  
Дочерний процесс обрабатывает команды и возвращает результат родителю.

\subsection*{Архитектура программы}

\dirtree{%
.1 Lab3/.
.2 inc/.
.3 ipc\_shm.h.
.3 parser.h.
.3 signals.h.
.2 src/.
.3 ipc\_shm.c.
.3 main.c.
.3 parser.c.
.3 signals.c.
.2 Makefile.
}

\vspace{5mm}

\section{Описание программы}

\subsection*{Файл \texttt{main.c}}

\textbf{Назначение:}  
Главный модуль программы, отвечающий за создание процессов, настройку сигналов, и взаимодействие через разделяемую память.

\textbf{Функции и логика:}
\begin{itemize}
    \item Инициализирует разделяемую память с помощью \texttt{ipc\_shm\_init()}.
    \item Создаёт дочерний процесс через \texttt{fork()}.
    \item Устанавливает обработчики сигналов для \texttt{SIGUSR1} и \texttt{SIGUSR2}.
    \item Родительский процесс:
    \begin{itemize}
        \item Считывает ввод пользователя.
        \item Записывает команду в разделяемую память.
        \item Посылает дочернему процессу сигнал \texttt{SIGUSR1}.
        \item Ожидает сигнал \texttt{SIGUSR2}, означающий готовность ответа.
        \item Выводит результат, прочитанный из общей памяти.
    \end{itemize}
    \item Дочерний процесс:
    \begin{itemize}
        \item После получения сигнала \texttt{SIGUSR1} извлекает команду.
        \item Передаёт её модулю парсинга \texttt{parse\_command()}.
        \item Записывает полученный результат обратно в shared memory.
        \item Посылает родителю сигнал \texttt{SIGUSR2}.
    \end{itemize}
    \item После завершения работы очищает ресурсы shared memory.
\end{itemize}


\subsection*{Файл \texttt{ipc\_shm.c}}

\textbf{Назначение:}  
Реализует операции работы с POSIX shared memory.

\textbf{Функции:}
\begin{itemize}
    \item \texttt{ipc\_shm\_init()} — создаёт или открывает сегмент shared memory, задаёт размер, отображает его в адресное пространство.
    \item \texttt{ipc\_shm\_get()} — возвращает указатель на структуру общей памяти (команда, ответ, флаг).
    \item \texttt{ipc\_shm\_destroy()} — отсоединяет память и удаляет объект.
\end{itemize}


\subsection*{Файл \texttt{parser.c}}

\textbf{Назначение:}  
Анализирует текстовые команды и выполняет вычисления.

\textbf{Функции:}
\begin{itemize}
    \item \texttt{parse\_command()} — определяет тип команды и вызывает нужный обработчик.
    \item Обработчики:
    \begin{itemize}
        \item \texttt{cmd\_add()}, \texttt{cmd\_sub()}, \texttt{cmd\_mul()}, \texttt{cmd\_div()} — арифметические операции.
        \item \texttt{cmd\_reverse()} — разворот строки.
        \item \texttt{cmd\_help()} — вывод списка команд.
    \end{itemize}
    \item Формирует строку результата для передачи родителю.
\end{itemize}


\subsection*{Файл \texttt{signals.c}}

\textbf{Назначение:}  
Устанавливает обработчики сигналов и определяет реакцию процессов на сигналы.

\textbf{Функции:}
\begin{itemize}
    \item \texttt{install\_handler(signum, handler)} — универсальная установка обработчика сигнала.
    \item Обработчики:
    \begin{itemize}
        \item \texttt{on\_parent\_signal()} — обработка сигнала \texttt{SIGUSR2} родителем.
        \item \texttt{on\_child\_signal()} — обработка сигнала \texttt{SIGUSR1} дочерним процессом.
    \end{itemize}
\end{itemize}


\subsection*{Файл \texttt{commands.c}}

\textbf{Назначение:}  
Содержит низкоуровневые функции, выполняющие конкретные команды.

\textbf{Функции:}
\begin{itemize}
    \item Реализация математических операций.
    \item Операции со строками.
    \item Формирование сообщений об ошибках.
\end{itemize}

\subsection*{Файл \texttt{ipc\_shm.h}}

\textbf{Структура данных:}

\begin{verbatim}
typedef struct {
    char command[256];
    char response[256];
    volatile int flag; // 0 - свободно, 1 - команда готова, 2 - ответ готов
} shm_area_t;
\end{verbatim}

