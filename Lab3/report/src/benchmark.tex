\section{Результаты}

Разработанное решение представляет собой программу на языке C, реализующую взаимодействие двух процессов (родительского и дочернего) через разделяемую память и системные сигналы.  
Родительский процесс передаёт дочернему строковые команды, содержащие числа типа \texttt{float}, а дочерний процесс выполняет вычисление их суммы и возвращает результат через общий сегмент памяти.

Программа корректно обрабатывает произвольное количество вещественных чисел в строке, поддерживает синхронизацию при помощи сигналов \texttt{SIGUSR1} и \texttt{SIGUSR2}, а также обеспечивает защиту от ошибок записи и чтения через флаговую систему в shared memory.

В ходе проверки работы программы было подтверждено, что:
\begin{itemize}
    \item дочерний процесс корректно получает команды из общей памяти;
    \item парсер успешно извлекает и суммирует числа с плавающей точкой;
    \item результат надёжно передаётся обратно родительскому процессу;
    \item реакция на сигналы осуществляется в нужном порядке, исключая гонки данных;
    \item обработка ошибок системных вызовов (создание памяти, установка сигналов, \texttt{fork()}) работает штатно.
\end{itemize}

\subsection*{Пример взаимодействия}

При вводе строки:
\begin{verbatim}
12.5 3.4 8.1
\end{verbatim}

дочерний процесс вычисляет сумму:
\begin{verbatim}
24.0
\end{verbatim}

и передаёт её родителю, который выводит результат в стандартный поток.

\subsection*{Ключевые особенности}

\begin{enumerate}
    \item Программа использует POSIX разделяемую память для обмена данными между процессами.
    \item Синхронизация основана на механизме сигналов \texttt{SIGUSR1} (от родителя к ребёнку) и \texttt{SIGUSR2} (от ребёнка к родителю).
    \item Формат команд позволяет обрабатывать произвольное число вещественных значений.
    \item Парсер реализован как отдельный модуль, что облегчает расширение набора команд.
    \item Структура проекта разделена на каталоги \texttt{inc/} и \texttt{src/}, что упрощает поддержку и развитие кода.
    \item Реализована обработка ошибок всех критически важных системных вызовов (создание shared memory, \texttt{fork()}, установка обработчиков сигналов).
\end{enumerate}
