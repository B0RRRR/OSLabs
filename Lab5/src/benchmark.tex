\section{Результаты}

\subsection*{Результат вывода \texttt{strace} по лабораторной работе 1}
Трассировка \texttt{strace} показывает, что программа корректно создаёт дочерний процесс, 
организует межпроцессное взаимодействие через неименованный канал и выполняет синхронизацию с помощью ожидания завершения дочернего процесса. Все используемые системные вызовы применяются в
правильной последовательности и соответствуют логике работы лабораторной программы.

\subsection*{Результат вывода \texttt{strace} по лабораторной работе 2}
Вывод утилиты \texttt{strace} показывает, что программа запускается как один процесс, динамически выделяет память и создаёт 
несколько потоков внутри этого процесса для параллельных вычислений. 
Многопоточность реализуется через системные вызовы \texttt{clone3} и \texttt{futex}, 
а результаты работы корректно записываются в файл, что подтверждает правильное использование 
библиотеки \texttt{pthread}.

\subsection*{Результат вывода \texttt{strace} по лабораторной работе 3}
Результаты трассировки с помощью \texttt{strace} показывают, что программа корректно загружается в адресное 
пространство процесса, инициализирует виртуальную память и стандартные библиотеки, после чего переходит к 
выполнению пользовательского кода. Завершение работы происходит штатно с выводом диагностического сообщения, 
что подтверждает корректную обработку ошибок и взаимодействие программы с операционной системой.

\subsection*{Результат вывода \texttt{strace} по лабораторной работе 4}
Анализ системных вызовов \texttt{strace} подтвердил фундаментальное различие в механизмах загрузки: 
в то время как системные библиотеки (\texttt{libc.so.6}) загружаются системным загрузчиком автоматически, 
\texttt{prog2} демонстрирует \textbf{позднюю линковку}, программно инициируя вызовы \texttt{openat} и \texttt{mmap} 
для загрузки \texttt{liblib1.so} уже в процессе выполнения.