\section{Выводы}

В ходе выполнения цикла лабораторных работ были успешно освоены фундаментальные механизмы системного 
программирования и взаимодействия программ с операционной системой. Анализ системных вызовов \texttt{strace} 
подтвердил освоение различных форм параллельной обработки: \textbf{межпроцессное взаимодействие} (ЛР 1) было 
реализовано через системные вызовы \texttt{fork}, \texttt{pipe} и \texttt{wait} для создания процессов и 
синхронизации через каналы; в то время как \textbf{многопоточность} (ЛР 2) использовала низкоуровневые вызовы 
\texttt{clone3} и \texttt{futex} для управления потоками внутри единого адресного пространства. 
Одновременно были изучены методы управления кодом и его загрузкой (ЛР 4). 
Был продемонстрирован контраст между автоматической загрузкой системных библиотек (\texttt{libc.so.6}) 
системным загрузчиком и программным управлением \textbf{поздней линковкой}, 
где тестовая программа \texttt{prog2} инициировала вызовы \texttt{openat} и \texttt{mmap} 
для загрузки \texttt{liblib1.so} уже во время выполнения.  Это подтверждает успешное освоение интерфейса \texttt{dlfcn} для реализации модульной архитектуры и плагинов. 
Наконец, базовая трассировка (ЛР 3) показала корректный процесс инициализации программы, включая выделение виртуальной памяти и штатную обработку системных зависимостей. 
Таким образом, был освоен полный цикл взаимодействия программы с ОС: от базовой загрузки и выделения ресурсов до реализации сложных механизмов параллелизма и динамического управления кодом.