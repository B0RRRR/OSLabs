\section{Условие}
{\bfseries Цель работы:}
Приобретение практических навыков в создании динамических библиотек и
в создании программ, которые используют функции динамических библиотек


{\bfseries Задание:}
Требуется создать динамические библиотеки, которые реализуют заданный вариантом функционал. 
Далее использовать данные библиотеки 2-мя способами:
1.	Во время компиляции (на этапе «линковки»/linking)
2.	Во время исполнения программы. Библиотеки загружаются в память с помощью интерфейса ОС для работы с динамическими библиотеками
В конечном итоге, в лабораторной работе необходимо получить следующие части:
Динамические библиотеки, реализующие контракты, которые заданы вариантом;
Тестовая программа (программа №1), которая используют одну из библиотек, используя информацию полученные на этапе компиляции;
Тестовая программа (программа №2), которая загружает библиотеки, используя только их относительные пути и контракты.
Провести анализ двух типов использования библиотек.



{\bfseries Вариант:} 28