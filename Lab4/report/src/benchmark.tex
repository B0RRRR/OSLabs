\section{Результаты}

Разработанное решение представляет собой систему из двух динамических библиотек (\texttt{liblib1.so}, \texttt{liblib2.so}) и двух исполняемых программ (\texttt{prog1}, \texttt{prog2}), демонстрирующих два метода линковки и загрузки кода.

\subsection*{Проверка Программы №1 (Ранняя Линковка)}
Было подтверждено, что программа \texttt{prog1} была успешно слинкована с \texttt{liblib1.so} на этапе компиляции.
\begin{itemize}
    \item Вызовы функций \texttt{Pi(K)} и \texttt{Square(A, B)} всегда используют \textbf{Реализацию 1} (Ряд Лейбница и Площадь прямоугольника).
    \item Присутствие \texttt{liblib1.so} является критически важным для запуска программы (\textbf{сильная зависимость}).
\end{itemize}

\subsection*{Проверка Программы №2 (Поздняя Линковка)}
Было подтверждено, что программа \texttt{prog2} успешно использует системный интерфейс \texttt{dlfcn.h} для динамической загрузки библиотек.
\begin{itemize}
    \item Программа корректно загружает \texttt{liblib1.so} при старте.
    \item Команда \texttt{0} успешно выполняет выгрузку текущей библиотеки и загрузку \texttt{liblib2.so} (или наоборот), демонстрируя \textbf{переключение контракта} во время исполнения.
    \item После переключения функции \texttt{Pi()} и \texttt{Square()} демонстрируют новое поведение (\textbf{Реализация 2}: Формула Валлиса и Площадь прямоугольного треугольника).
\end{itemize}


\subsection*{Пример взаимодействия}

Ниже приведен пример сессии взаимодействия с \textbf{Программой №2}, демонстрирующий смену реализации во время работы.

\textbf{Сессия:}
\begin{lstlisting}[language=bash, caption={Сессия работы с prog2}]
$ ./prog2
Dynamic Loading
[...]
Succes to load Program 1.

>>> 2 10 5
Square(10.00, 5.00) [Rectangle] = 50.00

>>> 0
Succes to load Program 2.

>>> 2 10 5
Square(10.00, 5.00) [Rect triangle] = 25.00

>>> 1 1000
Pi(1000) [Walis] = 3.14159049

>>> exit
\end{lstlisting}

\subsection*{Ключевые особенности}

\begin{enumerate}
    \item \textbf{Инкапсуляция Реализаций:} Разработаны две полностью независимые динамические библиотеки (\texttt{liblib1.so}, \texttt{liblib2.so}), каждая из которых реализует один и тот же контракт, что позволяет легко менять функционал без изменения основного кода.
    \item \textbf{Демонстрация Ранней Линковки:} Программа \texttt{prog1} подтверждает механизм \textbf{статической (ранней)} линковки, где зависимость разрешается загрузчиком ОС при запуске.
    \item \textbf{Динамическая Загрузка:} Программа \texttt{prog2} использует функции \texttt{dlopen}, \texttt{dlsym} и \texttt{dlclose}, демонстрируя \textbf{позднюю} линковку и возможность управления модулями из кода приложения.
    \item \textbf{Переключение Контракта:} Реализована команда \texttt{0}, которая обеспечивает смену всей логики вычислений (Pi и Square) во время исполнения, демонстрируя возможности архитектуры плагинов.
    \item \textbf{Независимость Кода:} Исходный код \texttt{main\_dynamic.c} не содержит ссылок на конкретные библиотеки, что делает его гибким и независимым от выбранной реализации.
\end{enumerate}