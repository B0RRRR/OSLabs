\section{Выводы}

В ходе выполнения лабораторной работы были успешно освоены методы создания и
использования динамических библиотек на языке С в среде Linux.
Были разработаны две независимые динамические библиотеки (\texttt{liblib1.so} и \texttt{liblib2.so}), реализующие единый контракт (\texttt{contract.h}),
что позволило обеспечить возможность замены функционала. На практике был изучен механизм ранней (статической) линковки (\texttt{prog1}),
при которой зависимость от \texttt{liblib1.so} фиксируется на этапе сборки и управляется системным загрузчиком, демонстрируя сильную зависимость
исполняемого файла от присутствия библиотеки. Параллельно был освоен механизм поздней (динамической) загрузки с использованием интерфейса \texttt{dlopen}/\texttt{dlsym}/\texttt{dlclose} в Программе №2.
Эта программа продемонстрировала гибкость и слабую зависимость, самостоятельно управляя жизненным циклом библиотек.
Ключевой результат был достигнут в Программе №2, где реализованная команда \texttt{0} позволила на лету переключать всю логику вычислений (Pi и Square) между Реализацией 1 и Реализацией 2.
Полученные результаты подтверждают, что поздняя линковка является мощным инструментом для создания модульных и расширяемых приложений (систем плагинов), где функционал может быть изменен или обновлен без необходимости перекомпиляции основного исполняемого файла.
Структура проекта, разделенная на контракты, реализации и тестовые программы,
обеспечила ясное разделение ответственности между модулями.