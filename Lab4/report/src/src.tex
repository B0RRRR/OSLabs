\section{Метод решения}

Задача состояла в создании динамических библиотек, реализующих заданный функционал (Вариант 28), и создании двух тестовых программ, демонстрирующих разные методы использования этих библиотек.

\subsection*{Контракты}

Для обеспечения совместимости между всеми модулями (двумя библиотеками и двумя программами) был разработан единый контракт, определенный в файле \texttt{contract.h}.

\begin{enumerate}
    \item \textbf{Функция 1 (Рассчет $\pi$):} \texttt{float Pi(int K)}
    \begin{itemize}
        \item Реализация 1 (\texttt{liblib1.so}): \textbf{Ряд Лейбница}.
        \item Реализация 2 (\texttt{liblib2.so}): \textbf{Формула Валлиса}.
    \end{itemize}
    \item \textbf{Функция 2 (Площадь):} \texttt{float Square(float A, float B)}
    \begin{itemize}
        \item Реализация 1 (\texttt{liblib1.so}): Площадь \textbf{прямоугольника} ($A \cdot B$).
        \item Реализация 2 (\texttt{liblib2.so}): Площадь \textbf{прямоугольного треугольника} ($\frac{1}{2} A \cdot B$).
    \end{itemize}
\end{enumerate}

\subsection*{Методы Использования Библиотек}

\begin{enumerate}
    \item \textbf{Программа №1 (Ранняя Линковка / Static Linking):} Исполняемый файл \texttt{prog1} линкуется с \texttt{liblib1.so} на этапе компиляции с использованием флага \texttt{-l}. Зависимость фиксируется в заголовке программы.
    \item \textbf{Программа №2 (Поздняя Линковка / Dynamic Loading):} Исполняемый файл \texttt{prog2} использует системный интерфейс (функции \texttt{dlopen}, \texttt{dlsym}, \texttt{dlclose} из \texttt{dlfcn.h} с линковкой \texttt{-ldl}) для загрузки и выгрузки \texttt{liblib1.so} или \texttt{liblib2.so} непосредственно во время работы программы.
\end{enumerate}


\subsection*{Архитектура программы и Структура файлов}

\dirtree{%
.1 lab4/.
.2 inc/.
.3 contract.h (Общий контракт).
.2 src/.
.3 lib1.c (Реализация 1: Лейбниц, Прямоугольник).
.3 lib2.c (Реализация 2: Валлис, Треугольник).
.3 main\_static.c (Тестовая программа №1: Ранняя линковка).
.3 main\_dynamic.c (Тестовая программа №2: Поздняя линковка).
.2 Makefile (Скрипт сборки).
}

\vspace{5mm}

\section{Описание Программных Модулей}

\subsection*{Файл \texttt{inc/contract.h}}

\textbf{Назначение:} Определяет сигнатуры функций \texttt{Pi(int K)} и \texttt{Square(float A, float B)}, создавая общий интерфейс, который должны реализовать обе динамические библиотеки.

\subsection*{Файл \texttt{src/lib1.c} (Реализация 1)}

\textbf{Назначение:} Реализует функции контракта, используя Ряд Лейбница для расчета $\pi$ и формулу площади прямоугольника.

\textbf{Особенности сборки:} Компилируется с флагами \texttt{-fPIC} (Position Independent Code) и \texttt{-shared} для создания динамической библиотеки \texttt{liblib1.so}.

\subsection*{Файл \texttt{src/lib2.c} (Реализация 2)}

\textbf{Назначение:} Реализует функции контракта, используя Формулу Валлиса для расчета $\pi$ и формулу площади прямоугольного треугольника.

\textbf{Особенности сборки:} Компилируется с флагами \texttt{-fPIC} и \texttt{-shared} для создания динамической библиотеки \texttt{liblib2.so}.

\subsection*{Файл \texttt{src/main\_static.c} (Программа №1)}

\textbf{Назначение:} Тестовая программа, демонстрирующая использование библиотеки через \textbf{раннюю линковку}.

\textbf{Логика:}
\begin{itemize}
    \item Осуществляет прямой вызов функций \texttt{Pi()} и \texttt{Square()}.
    \item Программа \texttt{prog1} при сборке линкуется с \texttt{liblib1.so} и всегда использует только эту реализацию.
    \item Ввод команд организован по протоколу: \texttt{1 arg1...} для \texttt{Pi}, \texttt{2 arg1 arg2...} для \texttt{Square}.
\end{itemize}

\subsection*{Файл \texttt{src/main\_dynamic.c} (Программа №2)}

\textbf{Назначение:} Тестовая программа, демонстрирующая использование библиотеки через \textbf{позднюю линковку} и переключение реализаций.

\textbf{Логика:}
\begin{itemize}
    \item Использует системный API: \texttt{dlopen} (загрузка библиотеки), \texttt{dlsym} (получение адреса функции) и \texttt{dlclose} (выгрузка).
    \item Вызов функций \texttt{Pi} и \texttt{Square} осуществляется через указатели на функции, полученные из \texttt{dlsym}.
    \item Команда \textbf{\texttt{0}} реализована для выгрузки текущей библиотеки и загрузки альтернативной, обеспечивая динамическую смену функционала во время исполнения.
\end{itemize}
