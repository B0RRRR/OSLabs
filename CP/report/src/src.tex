\section{Метод Решения }

Задача состояла в создании многопользовательской версии консольной игры "Быки и Коровы" с архитектурой Клиент-Сервер, реализованной на языке C++ с использованием Berkeley Sockets. Ключевая особенность решения — возможность динамического задания количества игроков при запуске сервера и обеспечение надежного, пошагового игрового процесса.


Архитектура представляет собой классическую модель Клиент-Сервер:
\begin{itemize}
    \item \textbf{Сервер (\texttt{server}):} Отвечает за прием подключений, управление очередью ходов, хранение секретного кода, расчет результатов ("Быки" и "Коровы") и рассылку служебных сообщений.
    \item \textbf{Клиент (\texttt{client}):} Предоставляет пользовательский интерфейс для ввода ходов и отображает результаты, полученные от сервера.
\end{itemize}
Взаимодействие осуществляется через TCP-сокеты.

\subsection*{Сетевой Контракт (Протокол)}

Для обеспечения совместимости между Клиентом и Сервером разработан простой текстовый протокол. Все сообщения состоят из команды и, при необходимости, аргументов, разделенных пробелом.

\begin{enumerate}
    \item \textbf{Идентификация:} \texttt{YOUR\_ID [ID]} \\
    \textit{Назначение:} Сервер сообщает клиенту его уникальный сокет-дескриптор, который будет использоваться в качестве ID игрока для управления очередью.
    
    \item \textbf{Ожидание:} \texttt{WAIT [N]} \\
    \textit{Назначение:} Сообщение клиенту о том, сколько еще игроков необходимо для начала игры.
    
    \item \textbf{Ход:} \texttt{TURN [ID]} \\
    \textit{Назначение:} Сервер информирует всех клиентов о том, чей сейчас ход. Клиент, чей ID совпадает с \texttt{[ID]}, получает разрешение на ввод.
    
    \item \textbf{Результат:} \texttt{RESULT [Bulls] [Cows]} \\
    \textit{Назначение:} Сервер сообщает результат последней попытки.
    
    \item \textbf{Победа:} \texttt{WIN [ID]} \\
    \textit{Назначение:} Объявление победителя.
    
    \item \textbf{Завершение:} \texttt{SERVER\_SHUTDOWN} \\
    \textit{Назначение:} Команда для корректного завершения работы клиента (высылается при остановке сервера или таймауте).
\end{enumerate}

\subsection*{Методы Обеспечения Динамичности и Надежности}

\begin{enumerate}
    \item \textbf{Настраиваемое количество игроков:} Количество необходимых для старта игроков (\texttt{players\_required}) задается аргументом командной строки при запуске сервера (\texttt{./server N}).
    \item \textbf{Обработка многозадачности (\texttt{select}):} Сервер использует функцию \texttt{select()} для неблокирующего ожидания подключений и ходов от текущего игрока, а также для контроля глобального флага остановки (\texttt{keep\_running}) по сигналу \texttt{SIGINT}.
    \item \textbf{Таймауты:} Введен таймаут на ход (60 секунд), по истечении которого сервер принудительно завершает игру, предотвращая зависание сессии из-за бездействия клиента.
\end{enumerate}

\subsection*{Архитектура программы и Структура файлов}

\dirtree{%
.1 lab4\_network\_game/.
.2 inc/.
.3 game.h (Интерфейс игровой логики).
.2 src/.
.3 game.cpp (Реализация игровой логики: генерация 3-значного кода, расчет Быков/Коров).
.3 server.cpp (Сетевой модуль: управление сессией, очередью, обработка сокетов).
.3 client.cpp (Клиентский модуль: пользовательский ввод/вывод, сетевое взаимодействие).
.2 Makefile (Скрипт сборки).
}

\vspace{5mm}

\section{Описание Программных Модулей}

\subsection*{Файл \texttt{inc/game.h} и \texttt{src/game.cpp}}

\textbf{Назначение:} Инкапсуляция всей игровой логики.

\textbf{Ключевые особенности:}
\begin{itemize}
    \item Установлена константа \texttt{CODE\_LENGTH = 3}, что делает секретное число \textbf{трехзначным}.
    \item Реализована функция \texttt{generate\_code()} для создания уникального 3-значного числа.
    \item Функция \texttt{process\_guess()} выполняет сравнение хода игрока с секретным кодом, возвращая количество Быков и Коров. Условием победы является \textbf{3 Быка}.
\end{itemize}

\subsection*{Файл \texttt{src/server.cpp}}

\textbf{Назначение:} Центральный узел управления игрой.

\textbf{Логика:}
\begin{itemize}
    \item Принимает количество игроков в качестве аргумента командной строки.
    \item Использует прослушивающий сокет и цикл `while` для приема подключений.
    \item Управляет вектором клиентских сокет-дескрипторов (\texttt{clients}).
    \item Внутри игрового цикла перебирает клиентов по очереди (оператор \texttt{for} по вектору \texttt{clients}).
    \item Использует \texttt{select} для блокирующего ожидания ввода от текущего игрока с установленным таймаутом.
\end{itemize}

\subsection*{Файл \texttt{src/client.cpp}}

\textbf{Назначение:} Интерфейс пользователя.

\textbf{Логика:}
\begin{itemize}
    \item Устанавливает соединение с сервером.
    \item Сохраняет присвоенный сервером \texttt{server\_client\_id} для дальнейшей идентификации.
    \item Работает в бесконечном цикле \texttt{while} ожидания сообщений от сервера (\texttt{recv}).
    \item **Блокировка ввода:** Ввод хода разрешен только при получении команды \texttt{TURN} с ID, совпадающим с \texttt{server\_client\_id}.
    \item Реализована команда \texttt{QUIT} для корректного выхода.
\end{itemize}