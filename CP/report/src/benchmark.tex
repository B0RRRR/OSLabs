\section{Результаты}

Разработанное сетевое решение было успешно протестировано, подтверждая корректность реализации архитектуры Клиент-Сервер, протокола взаимодействия и механизма динамической настройки игры.

\subsection*{Проверка Архитектуры Клиент-Сервер}
Было подтверждено, что сетевое взаимодействие основано на протоколе TCP, и сервер корректно обрабатывает несколько одновременных подключений.

\begin{itemize}
    \item \textbf{Идентификация:} Каждый клиент при подключении успешно получает от сервера уникальный \texttt{YOUR\_ID} (сокет-дескриптор, например, 4, 5, 6), который используется для управления очередью ходов.
    \item \textbf{Старт игры:} Сервер корректно ждет заданное количество игроков (\texttt{players\_required}) и только после этого рассылает команду \texttt{START}.
    \item \textbf{Зависимость:} Клиентский код не содержит логики игры, а полностью зависит от сервера, что подтверждает архитектуру тонкого клиента.
\end{itemize}


\subsection*{Проверка Механизма Управления Сессией и Очередностью}
Протестирован игровой цикл с использованием двух и более клиентов.

\begin{itemize}
    \item \textbf{Очередность ходов:} Сервер строго следует последовательности, рассылая команду \texttt{TURN [ID]} в порядке, соответствующем вектору подключенных клиентов.
    \item \textbf{Блокировка ввода (Клиент):} Клиентский модуль корректно блокирует пользовательский ввод, пока не получит команду \texttt{TURN} со своим \texttt{ID}, предотвращая преждевременную отправку данных (решена проблема из ранних этапов разработки).
    \item \textbf{Проверка логики игры:} Ввод трехзначных чисел корректно обрабатывается модулем \texttt{Game}, и результаты (\texttt{RESULT Bulls Cows}) точно соответствуют правилам игры.
\end{itemize}

\subsection*{Проверка Динамической Настройки и Надежности}
Была проверена способность системы адаптироваться к изменяющимся условиям.

\begin{itemize}
    \item \textbf{Динамическое количество игроков:} Запуск сервера с аргументом \texttt{./server 3} успешно заставил сервер ожидать трех клиентов, а цикл ходов корректно переключался между всеми тремя игроками.
    \item \textbf{Обработка таймаута (\texttt{select}):} При бездействии текущего игрока в течение 60 секунд срабатывает таймаут, сервер корректно завершает сессию, рассылая команду \texttt{SERVER\_SHUTDOWN} всем остальным клиентам.
    \item \textbf{Корректное завершение:} Команда \texttt{QUIT} от клиента или сигнал \texttt{SIGINT} (Ctrl+C) на сервере приводят к контролируемой остановке и закрытию всех сокетов.
\end{itemize}

\subsection*{Пример взаимодействия}

Ниже приведен пример сессии взаимодействия с двумя клиентами, демонстрирующий строгую очередность ходов и корректный расчет результата для трехзначного кода.

\textbf{Дано:} Секретный код сервера: \texttt{359}. Требуется 2 игрока.

\textbf{Сессия Клиента 1 (ID 4):}
\begin{lstlisting}[language=bash, caption={Сессия работы Клиента 1}]
$ ./client
Connected! Waiting for server ID...
Server assigned you ID: 4
Server: Waiting for 1 more player(s)...

--- GAME STARTED ---
>>> YOUR TURN (4) <<<
Enter your guess (3 unique digits) or QUIT: 123
Result: Bulls: 0, Cows: 1  // 3 on 3rd pos
Waiting for Player 5's guess...
Result: Bulls: 0, Cows: 1

>>> YOUR TURN (4) <<<
Enter your guess (3 unique digits) or QUIT: 567
Result: Bulls: 0, Cows: 1  // 5 on 1st pos
Waiting for Player 5's guess...
Result: Bulls: 0, Cows: 1
\end{lstlisting}

\textbf{Сессия Клиента 2 (ID 5):}
\begin{lstlisting}[language=bash, caption={Сессия работы Клиента 2}]
$ ./client
Connected! Waiting for server ID...
Server assigned you ID: 5

--- GAME STARTED ---
Waiting for Player 4's guess...
Result: Bulls: 0, Cows: 1

>>> YOUR TURN (5) <<<
Enter your guess (3 unique digits) or QUIT: 890
Result: Bulls: 0, Cows: 1 // 9 on 9th pos
Waiting for Player 4's guess...
Result: Bulls: 0, Cows: 1

Waiting for Player 4's guess...
Result: Bulls: 0, Cows: 1

>>> YOUR TURN (5) <<<
Enter your guess (3 unique digits) or QUIT: 359
Result: Bulls: 3, Cows: 0  // Win!

!!! GAME OVER !!! Player 5 won the game!
\end{lstlisting}

\subsection*{Ключевые особенности разработанного решения}

\begin{enumerate}
    \item \textbf{Настраиваемая Сессия:} Возможность задать количество игроков (\texttt{N}) при запуске сервера, что обеспечивает гибкость и повторное использование кода для разного числа участников.
    \item \textbf{Контроль Ввода/Вывода (\texttt{select}):} Использование системного вызова \texttt{select()} обеспечивает эффективное управление сокетными дескрипторами и позволяет серверу одновременно отслеживать ожидание подключения и таймаут хода.
    \item \textbf{Тонкий Клиент / Строгий Протокол:} Разделение логики: игровая логика полностью инкапсулирована в классе \texttt{Game} на сервере, а клиент выступает только в роли интерфейса, следующего четкому текстовому протоколу.
    \item \textbf{Обработка исключений:} Реализованы механизмы корректного завершения работы (\texttt{QUIT}, \texttt{SIGINT}, Таймаут) с рассылкой широковещательных сообщений \texttt{SERVER\_SHUTDOWN}.
\end{enumerate}