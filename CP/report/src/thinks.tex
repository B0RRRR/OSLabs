\section{Заключение}

В результате выполнения курсовой работы была успешно разработана и протестирована многопользовательская сетевая игра "Быки и Коровы", 
основанная на архитектуре Клиент-Сервер и реализованная на C++ с использованием Berkeley Sockets. 
Было достигнуто полное разделение логики: сервер управляет всей игровой сессией, очередностью ходов и содержит инкапсулированную логику расчета Быков/Коров, 
в то время как клиент выступает в роли тонкого интерфейса. Все поставленные задачи были решены, включая обеспечение 
стабильного TCP-взаимодействия и возможность динамического задания количества игроков при запуске сервера. 
Для повышения эффективности игры длина секретного кода была сокращена до трех цифр. 
Ключевым достижением стало применение системного вызова \texttt{select()}, что обеспечило надежное управление сокетными дескрипторами, эффективную обработку таймаутов на ход и корректную реакцию на сигналы завершения. 
Использование четко определенного текстового протокола гарантирует структурированный обмен данными и общую надежность системы. 
В целом, созданное решение является функциональным, демонстрирует грамотное проектирование сетевых приложений и готово к масштабированию.

